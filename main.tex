\documentclass{article}
\usepackage[utf8]{inputenc}

\title{Corporização em Conversação Mediada por Computação}
\author{Rafael Silva}
\date{Outubro 2018}

\begin{document}

\maketitle

\section{Introdução}



\section{Corporização}

O corpo tem função fundamental para complementação do entendimento durante a conversação. Segundo \cite{}, "a coisa mais importante em comunicação é ouvir o que não está sendo dito".

Durante uma conversa, é possível compreender mais sobre os falantes e suas falas quando analisado seu corpo. De acordo com \cite{}, "o corpo só fala a verdade".

A importância da representação do corpo foi verificada e tratada muito antes das conversações mediadas por computação. A palavra avatar descende do termo hindu avatara que significa "descendente". "O conceito de avatara é provavelmente um desenvolvimento do mito antigo de que, através do poder criador de Maya, um deus poderia assumir qualquer forma..." segundo Vesna[3].

Já no dicionário da língua portuguesa Priberam, a definição relacionada à informática diz que avatar[4] é "ícone gráfico escolhido por um usuário para o representar em determinados jogos e comunidades virtuais".
A Corporização nos Meios de Conversação Mediada por Computação
A corporização nas conversações mediadas por computador tem alguns aspectos fundamentais para o entendimentos entre os pares.

Neste artigo serão exibidos alguns exemplos de corporização em modelos de conversação mediada por computação que são utilizados atualmente e outros exemplos conceituais para estudo do corpo na conversação mediada por computação.
Chat
No bate-papo, não há representação de corpo. As ações do usuário são representadas por textos sendo exibidos na parte inferior da tela, na maioria das vezes. Alguns chats tentam encobrir esta falta da representação do corpo com outros elementos, como: representação de intenção (figura IRC), uso de imagens (figura chat uol), representação de ações (figura whatsapp), entre outros.
Chat Circles

VIEGAS e XXX em seu projeto chamado Chat Circles, buscam a corporização para representação de ações. [reler chat circles]
Releitura do Chat Circles
Durante o primeiro semestre de 2018 na disciplina de Comunicação Mediada por Computador, 
Jogos
Tagarelas
Conclusão
Intencionalidade
Posicionamento
Presença
ações (digitando)
Interesse (assunto a ser discutido)





http://www.teleduc.org.br/sites/default/files/publications/joeiras\_ihc2000.pdf
http://www.bocc.ubi.pt/pag/julio-bruno-identidade-interaccao-social.pdf



\end{document}
